\documentclass[11pt]{article}
\usepackage[T1]{fontenc}
\usepackage{textcomp}
\usepackage{lmodern}
\usepackage[margin=1in]{geometry}
\usepackage[pdftex]{graphicx, color}
\usepackage{mathtools}
\usepackage{listings}
\usepackage[pdfusetitle]{hyperref}
\usepackage{forest}

\newcommand\tab[1][1cm]{\hspace*{#1}}
\usepackage{array}
\newcommand{\PreserveBackslash}[1]{\let\temp=\\#1\let\\=\temp}
\newcolumntype{C}[1]{>{\PreserveBackslash\centering}p{#1}}
\newcolumntype{R}[1]{>{\PreserveBackslash\raggedleft}p{#1}}
\newcolumntype{L}[1]{>{\PreserveBackslash\raggedright}p{#1}}

\usepackage{tikz}
\usetikzlibrary{automata,positioning}

\let\epsilon\varepsilon
\tikzset{shorten >=1pt, node distance=2cm, on grid, baseline={([yshift=-8pt] current bounding box.north)}}


\lstset{basicstyle=\small\ttfamily,breaklines=true}

\title{CS143 Spring 2022 -- Written Assignment 2}

\begin{document}
\begin{center}
% Change this:
\LARGE YOURNAME -- SUNETID \\
\LARGE CS143 Spring 2022 -- Written Assignment 2 \\
\end{center}

This assignment covers context free grammars and parsing. You may discuss this assignment with other students and work on the problems together. However, your write-up should be your own individual work, and you should indicate in your submission who you worked with, if applicable. Assignments can be submitted electronically through Gradescope as a \textsc{pdf} by 11:59 \textsc{pm pdt}. Please review the the course policies for more information: \url{https://web.stanford.edu/class/cs143/policies/}. A \LaTeX{} template for writing your solutions is available on the course website.
If you need to draw parse trees in \LaTeX{}, you may use the {\tt forest} package: \url{https://ctan.org/pkg/forest}.

\bigskip

\begin{enumerate}
% Problem 1
\item  Give a context-free grammar (CFG) for each of the following languages. Any grammar is acceptable---including ambiguous grammars---as long as it has the correct language. The start symbol should be $S$.
  \begin{enumerate}
  \item The set of all strings over the alphabet $\{1,2,*\}$ representing valid products of integers where the expression evaluates to some even value.

    Example Strings in the Language:
    \begin{center}
      112 \tab \tab 2*121 \tab \tab 221*1*1122
    \end{center}
    Strings not in the Language:
    \begin{center}
      $\epsilon$ \tab \tab 11*121  \tab \tab 12**22*112
    \end{center}
    
    \textbf{Solution}:
    % even * even , odd * even , even * odd , even
    $$S\to S*S\ |\ T*S\ |\ S*T\ |\ N2\ |\ 2$$
    % even = any number followed by a 2
    % $$E\to 2\ |\ N2$$
    % any number
    $$N\to 2\ |\ 1\ |\ N1\ |\ N2$$
    % odd = any number followed by 1 or odd*odd
    $$T\to 1\ |\ N1\ |\ T*T$$
    
  \item The set of all strings over the alphabet $\{x,(,),;\}$ representing nested tuples of $x$'s where each tuple has an even length.

    Example Strings in the Language:
    \begin{center}
      $()$ \tab \tab $(x;())$ \tab \tab $((); x; ((); x); x)$
    \end{center}
    Strings not in the Language:
    \begin{center}
      $\epsilon$ \tab \tab $x$ \tab \tab $((); x; x)$  \tab \tab $(x; (); (x; (); x); x)$
    \end{center}
    
    \textbf{Solution}:
    $$S\to ()\ |\ (T)$$
    $$T\to S\ |\ x;x\ |\ S;x\ |\ x;S\ |\ S;S\ |\ T;T\ $$
    
  \item  The set of all strings over the alphabet $\{0, 1\}$ where the number of $1$'s is exactly one more than the number of $0$'s.

    Example Strings in the Language:
    \begin{center}
      1 \tab \tab 101  \tab \tab 001110101
    \end{center}
    Strings not in the Language: 
    \begin{center}
      $\epsilon$ \tab \tab 001 \tab \tab 01100101
    \end{center}
    
    \textbf{Solution}:
    % balanced 0&1s concat with string with exactly one more 1
    % ST in covered cuz same string can be interpreted as both TS and ST
    $$S\to 1T\ |\ TS$$
    $$T\to \epsilon\ |\ 1T0\ |\ 0T1\ |\ TT$$
    
  \item  The set of all strings over the alphabet $\{0,1\}$ in the language $L:\{0^i 1^j 0^k \mid j \neq i+k\}$.

    Example Strings in the Language: 
    \begin{center}
      00 \tab \tab 110000  \tab \tab 000111110
    \end{center}
    Strings not in the Language: 
    \begin{center}
      $\epsilon$ \tab \tab 0011 \tab\tab 01111000
    \end{center}

    \textbf{Solution}:
    % let X = balanced 0&1s ends with 1, Y = balanced 0&1s starts with 1
    % target = X<[1]+>Y or <[0]+>XY or XY<[0]+> or <[0]+>XY<[0]+>
    $$S\to XAY\ |\ BXY\ |\ XYB\ |\ BXYB$$
    $$A\to 1\ |\ 1A$$
    $$B\to 0\ |\ 0B$$
    $$X\to \epsilon\ |\ 0X1$$
    $$Y\to \epsilon\ |\ 1Y0$$
  \end{enumerate}

  \newpage

% Problem 2
\item Consider the following grammar for if-then-else expressions that involve a variable $x$:
  \begin{align*}
    E &\to \text{if} \ x \ \text{then} \ E \mid \mathit{M} \\
    \mathit{M} &\to \text{if} \ x \ \text{then} \ \mathit{M} \ \text{else} \ E \mid x
  \end{align*}
  Is this grammar ambiguous or not?
  If yes, give an example of an expression with two different parse trees and draw the two parse trees.
  If not, explain why that is the case.

  \textbf{Solution}:
  % intuition: similar to classic "if-then-else ambiguity", think about "if x then if x then x else x". This works when parsing the "if x then x else x" (latter part) to M in one parse tree, but doesn't work when one tries to parse the "if x then x" in the middle as the first step. Cuz one has to parse the rest as M -> if x then (if x then x) else E, but (if x then x) cannot be parsed as M as in the production rule, hence it fails.
  % improve our first guess: how about changing the middle part (if x then x) to "if x then x else x" ? Then the "else x" at the end can't contribute to ambiguity. Making a second guess: "if x then x else if x then x", the "else x" should be fine this time. Whole sentence = "if x then if x then x else if x then x else x"
  \\
  It is ambiguous. Example: "if x then if x then x else if x then x else x"
  \begin{center}
  \begin{tikzpicture}
  \node {$E$}
    child {node {if}}
    child {node {$x$}}
    child {node {then}}
    child {node {$E$}
        child {node {$M$}
            child {node {if}}
            child {node {$x$}}
            child {node {then}}
            child {node {$M$}
                child {node {$x$}}
            }
            child {node {else}}
            child {node {$E$}
                child {node {$M$}
                    child {node {if}}
                    child {node {$x$}}
                    child {node {then}}
                    child {node {$M$}
                        child {node {$x$}}
                    }
                    child {node {else}}
                    child {node {$E$}
                        child {node {$M$}
                            child {node {$x$}}
                        }
                    }
                }
            }
        }
    };
  \end{tikzpicture}
  \end{center}
  
  \vspace{1cm}
  
  \begin{center}
  \begin{tikzpicture}
  \node {$E$}
    child {node {$M$} [sibling distance = 3cm]
        child {node {if}}
        child {node {$x$}}
        child {node {then}}
        child {node {$M$} [sibling distance = 1.5cm]
            child {node {if}}
            child {node {$x$}}
            child {node {then}}
            child {node {$M$}
                child {node {$x$}}
            }
            child {node {else}}
            child {node {$E$} [sibling distance = 1cm]
                child {node {if}}
                child {node {$x$}}
                child {node {then}}
                child {node {$E$}
                    child {node {$M$}
                        child {node {$x$}}
                    }
                }
            }
        }
        child {node {else}}
        child {node {$E$} 
            child {node {$M$}
                child {node {$x$}}
            }
        }
    };
  \end{tikzpicture}
  \end{center}

  \newpage

% Problem 3
\item
  \begin{enumerate}
  \item Eliminate left recursion from the following grammar:
    \begin{equation*}
      \begin{split}
        S &\to S(T) \mid Sa \mid [ T ] \mid Tb \\
        T &\to T(S) \mid Tc \mid d
      \end{split}
    \end{equation*}

    \textbf{Solution}:
    $$S\to [T]S'\ |\ TbS'$$
    $$S'\to (T)S'\ |\ aS'\ |\ \epsilon$$
    $$T\to dT'$$
    $$T'\to (S)T'\ |\ cT'\ |\ \epsilon$$
    
  \item Left factor the following grammar:
    \begin{equation*}
      \begin{split}
        S &\to (T+T) \mid (T) \\
        T &\to U*T \mid U*? \mid [ U ] \\
        U &\to U0 \mid U1 \mid \epsilon 
      \end{split}
    \end{equation*}
    
    \textbf{Solution}:
    $$S\to (TS'$$
    $$S'\to +T)\ |\ )$$
    $$T\to U*T'\ |\ [U]$$
    $$T'\to T\ |\ ?$$
    $$U\to UU'\ |\ \epsilon$$
    $$U'\to 0\ |\ 1$$
  \end{enumerate}

  \newpage
  
% Problem 4
\item Consider the following CFG, where the set of terminals is $\{a, b, c, d, \# , ?\}$:
  \begin{equation*}
    \begin{split}
      S &\to \# U T \mid T ? \\
      T &\to aS \mid bUc \mid \epsilon \\
      U &\to aSc \mid bTd \\
    \end{split}
  \end{equation*}

  \begin{enumerate}
  \item Construct the FIRST sets for each of the nonterminals.
    
    \textbf{Solution}:
    \begin{itemize}
    \item $S: \{a, b, \#, ?\}$
    \item $T: \{a, b, \epsilon\}$
    \item $U: \{a, b\}$
    \end{itemize}
    
  \item Construct the FOLLOW sets for each of the nonterminals.

    \textbf{Solution}:
    \begin{itemize}
    \item $S: \{c, d, ?, \$\}$
    \item $T: \{c, d, ?, \$\}$
    \item $U: \{a, b, c, d, ?, \$\}$
    \end{itemize}

  \item Construct the LL(1) parsing table for the grammar.

    \textbf{Solution}:
    \begin{center}
      \begin{tabular}{c|C{3em}|C{3em}|C{3em}|C{3em}|C{3em}|C{3em}|C{3em}}
        Nonterminal & $a$ & $b$ & $c$ & $d$ & $\#$ & $?$ & $\$$ \\
        \hline
        $S$ & $T?$ & $T?$ & & & $\#UT$ & $T?$ & \\
        \hline
        $T$ & $aS$ & $bUc$ & $\epsilon$ & $\epsilon$ & & $\epsilon$ & $\epsilon$ \\
        \hline
        $U$ & $aSc$ & $bTd$ &  & & & & \\
      \end{tabular}
    \end{center}
    
  \item Show the sequence of stack, input and action configurations that occur during an LL(1) parse of the string ``$\# a?ca?$''. At the beginning of the parse, the stack should contain a single $S$.

    \textbf{Solution}:
    \begin{center}
      \begin{tabular}{R{10em}|R{10em}|L{10em}}
        Stack & Input & Action \\
        \hline
        $S\$$ & $\#a?ca?\$$ & output $S\to \#UT$ \\
        $\#UT\$$ & $\#a?ca?\$$ & match $\#$ \\
        $UT\$$ & $a?ca?\$$ & output $U\to aSc$ \\
        $aScT\$$ & $a?ca?\$$ & match $a$ \\
        $ScT\$$ & $?ca?\$$ & output $S\to T?$ \\
        $T?cT\$$ & $?ca?\$$ & output $T\to \epsilon$ \\
        $?cT\$$ & $?ca?\$$ & match $?$ \\
        $cT\$$ & $ca?\$$ & match $c$ \\
        $T\$$ & $a?\$$ & output $T\to aS$ \\
        $aS\$$ & $a?\$$ & match $a$ \\
        $S\$$ & $?\$$ & output $S\to T?$ \\
        $T?\$$ & $?\$$ & output $T\to \epsilon$ \\
        $?\$$ & $?\$$ & match $?$ \\
        $\$$ & $\$$ & accept \\
        & & \\
        & & \\
        & & \\
      \end{tabular}
    \end{center}
  \end{enumerate}

  \newpage

% Problem 5 
\item Consider the following grammar $G$ over the alphabet $\{a,b,c\}$:
  \begin{equation*}
    \begin{split}
      S' &\to S \\
      S &\to Aa \\
      S &\to Bb \\
      A &\to Ac \\
      A &\to \epsilon\\
      B &\to Bc \\
      B &\to \epsilon \\
    \end{split}
  \end{equation*}
  You want to implement $G$ using an SLR(1) parser. Note that we have already added the $S' \to S$ production for you.
  \begin{enumerate}
  \item Construct the first state of the LR(0) machine, compute the FOLLOW sets of A and B, and point out the conflicts that prevent the grammar from being SLR(1).

    \textbf{Solution}:

  \item Show modifications to production 4 ($A \to Ac$) and production 6 ($B \to Bc$) to make the grammar SLR(1) while having the same language as the original grammar $G$. Explain the intuition behind this result.

    \textbf{Solution}:
  \end{enumerate} 

\end{enumerate}
\end{document}
